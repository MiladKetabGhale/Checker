\documentclass{llncs}
\usepackage{bussproofs}
\usepackage{holtexbasic,url,amsmath,environ}
\renewcommand{\HOLTokenTurnstile}{\ensuremath{\vdash\!\!}}
\renewcommand{\HOLinline}[1]{\ensuremath{#1}}
\renewcommand{\HOLKeyword}[1]{\mathsf{#1}}
\renewcommand{\HOLConst}[1]{{\textsf{\upshape #1}}}
\renewcommand{\HOLTyOp}[1]{\textsf{\itshape #1}}
\renewcommand{\HOLSymConst}[1]{\HOLConst{#1}}
\renewcommand{\HOLTokenBar}{\ensuremath{\mathtt{|}}}
\NewEnviron{holthmenv}{
  \begin{equation*}
  \begin{array}[t]{l}
  \BODY
  \end{array}
  \end{equation*}}
\newcommand{\TODO}[1]{{\bf TODO:} #1}
\begin{document}
\title{Verified Certificate Checking for Counting Votes}
\author{
Milad~K.~Ghale\inst{1}\and
Dirk~Pattinson\inst{1}\and
Ramana~Kumar\inst{2}}
\institute{Australian National University\and
Data61, CSIRO and UNSW}
\maketitle
\begin{abstract}
We approach electronic vote counting as a certified computation where for each execution of the counting protocol, a certificate accompanies the end result. The certificate can then be independently checked for validity by a checker. We introduce a new framework for verifying electronic vote counting results that are based on the Single Transferable Vote scheme (STV). We demonstrate the process by formalizing an STV counting protocol and its certificates inside the theorem prover HOL. Then we obtain a verified checker inside HOL that scrutinizes formal certificates. By means of a built-in provably correct mechanism, we translate the checker from HOL into CakeML. Combining the translated checker, translated parser, and the I/O aspects inside CakeML environment, we prove that the combination performs the computation for checking real certificates correctly down to the machine code. Therefore, we achieve an extraordinary level of verification for deciding which election results are authentic. Besides requiring a low amount of labor, our formalization is modular in that it can be easily adapted to accommodate other STV schemes. Furthermore, the executable extracted from CakeML is capable of handling real size elections.

\end{abstract}
\section{Introduction}\label{sec:intro}

The main contribution of this paper is a new framework for
verifiably correct vote counting. 
%
Electronic voting is becoming more and more prevalent worldwide. 
But almost scandalously, 
the current state of affairs leaves much
to be
desired as the public vote is a cornerstone of
modern democracy.
Indeed electronic techniques as they are used now may be seen as a step back from traditional
paper based elections.

For example, the vote counting software that is used in Australia's
most populous 
state, the state of New South Wales, 
was found to contain errors that had an impact in at least one seat that
was wrongly filled with high probability.
This was not only reported in specialist publications
\cite{Conway:2017:ANS} but
also in the national press \cite{Brooks:2017:NCE}.
 
When counting ballots by hand,  the counting is monitored by
scrutineers, usually members of the general public or stakeholders
such as party representatives and members of the general public. 
%
In contrast,  computer software that is used to count ballots 
merely produces a final result. Moreover, in many cases, the source
code of these programs is commercial in confidence, and there is no
evidence of the correctness of the count that could be seen as
analogous to scrutineers in traditional, paper-based vote counting.

It is universally recognised that transparent and verifiable vote
counting is a key constituent to 
establish trustworthiness, and subsequently trust, in the final outcome. 
The computer-based methods 
currently in use fail to meet both expectations.  

In the literature on electronic voting, the notion of \emph{universal
verifiability} of vote counting (any voter can check that the
announced result is correct on the basis of the published ballots
\cite{Kremer:2010:EVE}) has long been recognised as being
central,
both for guaranteeing  correctness, and building trust,
in electronic elections. For vote-counting (on which we focus in
this paper), verifiability means that every stakeholder, or indeed
any
member of the general public, has the means to check that the
computation
of election winners is correct.

The approach presented here combines the concept of certifying
algorithms \cite{McConnell:2011:CA} with formal specification and
theorem proving to address this challenge. In a nutshell, a
\emph{certifying algorithm} is an algorithm that produces, with
every execution, an easily-verifiable certificate of the correctness
of the computed result.  This certificate can then be scrutinised by
a verifier, independently of the tools, hardware or software that
were used to create the certificate. 

Our focus in this paper is on the \emph{verifier}. We briefly
discuss a 
concise formal specification of \emph{single transferable vote}, a
complex, preferential voting system used e.g. in Ireland, Malta, New
Zealand and Australia. From this specification, we develop a notion
of \emph{certificate} so that correct certificates guarantee
correctness of election results. The main body of our work concerns
the \emph{verifier} (certificate checker), and we present a
synthesis of the verifier that is itself fully verified down to the
machine level. 


This proceeds in four steps. 
First, we formalise  the vote counting protocol as a sequence of
steps inside the HOL theorem prover where every step by and large
corresponds to an action taken by a counting officer in a
paper-based setting.  There are two kinds of stages that we call
\emph{judgements} in analogy to typing assertions in type theory.
\emph{Final} judgements decleare a set of winners. 
\emph{Non-final} judgements represent the current state of the count
as a multi-component tuple, corresponding to a snapshot of the stage
of the count in a paper-based setting.  The formalisation of the
voting protocol then takes the form of \emph{rules} that specify how
to advance from one judgement to the next, thereby advancing the
count. The applicability of particular rules are described 
by side conditions that are in turn formalised by HOL predicates.  A
correct certificate is then simply a sequence of judgements where
each judgement is justified through its predecessor by means of a
correct rule application.  The task of the verifier is then simply
to process a list of judgements and ascertain that this is indeed
the case.  In particular, our specification of rules is purely
logical.


Second, in tandem with the logical specification of each rule, we
define a boolean function that checks whether or not the rule has
been applied correctly.  This then directly gives rise to the
verifier that, at every step, just checks whether any of the rules
is applicable, using the corresponding boolean function. 

Third, we establish correspondence between the logical
definitions and their computational counterparts. This simply boils
down to formally establishing that the logical specification holds
if and only if the boolean function returns \emph{true}. This then
straightforwardly implies the correctness of the certificate
verifier. This allows us to conclude that a valid certificate indeed
implies that the election protocol has been carried out in
accordance to the specification.


In a forth, and last step, we synthesise an executable for the
verifier that is correct down to the machine level.  This is
achieved by automatically translating HOL-definitions into CakeML
the machine code. This guarantees that the ensuing CakeML functions
are coputationally equivalent to their HOL counterparts. 
To perform computation on an actual certificate, we define a formal
syntax of the latter, and a parser in HOL, that we combine with the
I/O mechanisms of CakeML to obtain the verifier. 
In particular, this allows us to prove correctness of the verifier
with respect to a high-level specification. 
HOL. This finally induces an executable the correctness of which is
verified down to the machine level. 

In summary, our slogan is ``verify the verifier''. Rather than
verifying the \emph{program} that performs an election count, we
demand that a program produces a certificate that we can then
independently verify. This has several advantages. For one, it is
much less labour intensive to verify the verifier, compared with
program verification. Second, having verifiable certificates at hand
meets the goal of universal verifiability \cite{Kremer:2010:EVE}.
Third, we achieve correctness over a minimal trust base through the
use of CakeML. 

In the remainder of the paper, we describe our framework in detail
and demonstrate that it can handle real-world size elections by
evaluating it on historical data of elections conducted in
Australia.



\noindent\textbf{Related Work.} Alkassar et al~\cite{EAlk} combine
certified computation and theorem proving with methods of code
verification to establish a framework for validation of certifying
algorithms. First they implement algorithms in C language. Then by
using the VCC tool~\cite{}, they obtain some pre- and
postconditions, which are next generalized syntatically and then
implemented in the Isabella theorem prover to be discharged. Even
though this metohd introduces a uniform platform for dealing with
certifiying algorithms, the user has to trust the VCC tool, and
there is duplication of effort in that one has to generalize the
conditions imposed by the VCC and then implement them manually in
Isabella to prove. To ameliorate this disadvantage, Noschinski et
al~\cite{LNos} replaced the intermediate step where VCC is invoked
by the AutoCorres~\cite{DGre} verifier which provably correctly
translates (part of) C language into Isabella in a semantic
preserving manner. Although the latter method is much superior in
the degree of the verification achieved, one can not guarantee that
the machine code behaves in the same way as its top-level C
implementation. On the other hand, in context of vote counting
verification, merely certifying implementations
exist~\cite{DPat,DPati,MKet}. For STV, in particular, Ghale et
al~\cite{MKet} and Pattinson et al~\cite{DPati} verify some STV
schemes inside Coq and output a certificate for each input upon
every execution of the extracted Haskell program. Nonetheless, they
do not provide any verified checker to independently scrutinize the
certificate for authenticity.  \end{document} \section{The Protocol
and Its HOL Formalization} STV is an election scheme used in
multi-seated constituencies and is designed to reflect the
proportional preferences of voters in that district for competing
candidates. The preference is expressed by ordering candidates on
each ballot. Then the count proceeds by assigning the value of the
ballot to the most preferred continuing candidate. A continuing
candidate is one who has neither been elected nor eliminated from
the initial list of competing candidates. When a candidate is not
continuing any more, they are not considered in the preferences of
ballots any further. Additionally, transfer of ballots of the
elected or eliminated candidates may occur until all vacancies are
filled.



There are various versions of STV. They mainly differ in how  and
when ballots are transferred or candidates are elected , the
transfer value of ballots of elected candidates, and the tie
breaking methods for determining which candidate is the weakest to
exclude. Here, we particularly deal with a STV called ANU\_Union. It
is similar to the STV used in the Australian Senate election but
with three distinctive properties: \begin{description}
\item[Step-by-step surplus transfer.] Surplus votes of already
elected candidates, who are awaiting for their surplus to be
transferred, are dealt with, one at a time, in order of first
preferences.  \item[Electing after each transfer.] After each
transfer of values, candidates that reach the quota are elected
immediately.  \item[Fractional transfer.] The value of vote transfer
is a fractional number determined by a specific
formula.\end{description}


The protocol proceeds as follows.  \begin{center} \begin{enumerate}
\item decide which ballots are \emph{formal}.  \item determine what
the quota exactly is.  \item count the first preference for each
\emph{formal} ballot paper and place the vote in the pile of the
votes of the preferred candidate.  \item if there are vacancies, any
candidate that reaches the quota is declared elected.  \item if all
the vacancies have been filled, counting terminates and the result
is announced.  \item if the number of vacancies exceeds the number
of continuing candidates, all of them are declared elected and the
result is announced.  \item if there are still vacancies and all
ballots are counted, and there is an elected candidate with surplus,
go to step $8$ otherwise go to step $9$.  \item In case of surplus
votes, transfer them to the next continuing preference appearing on
each of those votes at a fractional value according to the following
formula:

\begin{equation}  \label{eq:tv} \mbox{new value} =
\frac{\mbox{number of votes of elected candidate} -
\mbox{quota}}{\mbox{number of votes of elected candidate}}
\end{equation}

\noindent Subsequent transfer values are computed as the product of
the current value with previous transfer value.  \item if there are
still vacancies and all ballots are counted, and all surplus votes
are transferred, choose the candidate with the least amount of votes
and exclude them from the list of continuing candidates. Also
transfer all of their votes according to the next preference
appearing on each of the votes in his pile. The transfer value of
the ballots shall remain unchanged.  \item if there is more than one
elected candidate, first transfer the surplus of the candidate who
has the largest surplus. If after a transfer of surplus, a
continuing candidate exceeds the quota, declare them elected and
transfer their surplus, only after all of the earlier elected
candidates' surpluses have been dealt with.  \item at transfer
stage, candidates who are already elected or eliminated receive no
vote.  \end{enumerate} \end{center}

\subsection{An Actual Certificate Sample} Figure~\ref{fig;figure}
illustrates an example of an actual certificate for a small election
where three candidates $A$, $B$, and $C$ are the initial competing
candidates. The list of ballots $ba$ cast to the election consists
$b_1=([A,C],1)$, $b_2=([A,B,C],1)$, $b_3=([A,C,B],1)$,
$b_4=([B,A],1)$, $b_5=([C,B,A],1)$.  From the top, the first three
lines of the certificate are the header part visualising the quota
of the election as a fractional number, number of seats to be
filled, and the list of candidates initially participating in the
election. The fourth line depicts the terminal stage of the count
where list of final winners are declared. The remainder of the lines
show all of the intermediate steps taken from the starting state to
obtain the mention final state.  \begin{small} \begin{figure}[b]
\begin{tabular}{c@{\hspace{2cm}}c} \AxiomC{\tiny 133\%50} \noLine
\UnaryInfC{\tiny 2} \noLine \UnaryInfC{\tiny $[A,B,C]$} \noLine
\UnaryInfC{\scriptsize [A,C]} \RightLabel{\tiny hwin}
\UnaryInfC{\tiny []; A[3\%1] B[111\%100] C[133\%100]; A[] B[]
C[$b_{5}$,([C],11\%),([C,B],11\%100),([C],11\%100)]; []; [A]; [C]}
\RightLabel{\tiny count} \UnaryInfC{\tiny [$b_4$,([B,C],0.11)];
A[3\%1] B[111\%100] C[122\%100], A[] B[$b_4$,([B,C],11\%100)]
C[$b_{5}$,([C],11\%100),([C,B],11\%100)];[];[A];[C]}
\RightLabel{\tiny elim} \UnaryInfC{\tiny []; A[3\%1] B[111\%100]
C[122\%100], A[] B[$b_4$,([B,C],11\%100)]
C[$b_5$,([C],11\%100),([C,B],113\%1000)]; []; [A]; [B,C]}
\RightLabel{\tiny count} \UnaryInfC{\tiny
[([A,C],11\%100),([A,B,C],11\%100),([A,C,B],11\%100)]; A[3\%1]
B[1\%1] C[1\%1]; A[] B[$b_4$] C[$b_5$]; []; [A]; [B,C]}
\RightLabel{\tiny transfer} \UnaryInfC{\tiny []; A[3\%1] B[1\%1]
C[1\%1]; A[([A,C],11\%100),([A,B,C],11\%100),([A,C,B],11\%100)]
B[$b_4$] C[$b_5$]; [A]; [A]; [B,C]} \RightLabel{\tiny elect}
\UnaryInfC{\tiny []; A[3\%1] B[1\%1] C[1\%1]; A[$b_1$,$b_2$,$b_3$]
B[$b_4$] C[$b_5$]; []; []; [A,B,C]} \RightLabel{\tiny count}
\UnaryInfC{\tiny ba; A[0\%1] B[0\%1] C[0\%51]; A[] B[] C[]; []; [];
[A,B,C]} \DisplayProof \end{tabular} \caption{example of a
certificate} \label{fig;figure} \end{figure} \end{small}


As you see, the non-final stages consist of all necessary and
sufficient information for a scrutineer to independently verify the
counting process.  Every non-final line is constituted of six
components sepearated from one another by the semi-colon symbol. The
first component illustrates the ballots to be counted. Next is the
candidates with their tallies. After this, pile of ballots assigned
to each candidate is shown. The last three component depict the
backlog, the list of elected candidates, and the list of continuing
candidates, respectively. Here we describe how according to the
protocol counting steps are taken from the initial non-final stage,
which is the bottom-most line, to reach the final state. 



\textbf{count.} First preferences for each candidates are determined
and put into their pile. Hence candidate $A$ receives the ballots
$b_1$, $b_2$, and $b_3$. Candidate B receives the ballot $b_4$, and
C receives $b_5$. Tallies are updated so that tally of A becomes
$3$, and candidate B and C reach to $1$.\\ \textbf{elect.} Candidate
A exceeds the quota, he is elected and value of surplus votes
changes to $0.11$ according to the mentioned formula (\ref{eq:tv}).
The updated pile of A is $([A,C],0.11)$, $([A,B,C],0.11)$, and
$([A,C,B],0.11)$. Nothing changes about C and B at this stage.\\
\textbf{transfer.} As there are vacancies and no one else has
reached or exceeded the quota, surplus of A is dealt with. The list
of uncounted ballots is updated to contain the surplus of A. \\
\textbf{count.} The list of uncounted ballots is dealt with and
votes are distributed according to next continuing preference.
Therefore, C receives  two new votes (each of value $0.11$) which
are $([C],0.11)$ and $([C,B],0.113)$. Candidate B receives one vote,
which is $([B,C],0.11)$.\\ \textbf{elim.} No continuing candidate
has reached the quota, one vacancy is left, and there are no more
votes to deal with. So the weakest candidate is found and excluded,
which is B.\\ \textbf{count.} Candidate C receives the vote
$([c],0.11)$ from the excluded candidate B.\\ \textbf{hwin.} The
only continuing candidate, that is C, is elected and as we have
filled all the vacancies, a final stage has been obtained.


Such actual certificates are the printout counterparts of their
formal representatives. The connection between the actual and formal
certificate is established by a parser. In short, we explain how the
certificate is formalised and how the parser performs the parsing.
\subsection{The HOL Formalisation} The counting process is
constituted of some key components, all of which appear in a
comprehensive certificate of the count: \begin{enumerate} \item
candidates competing in the election \item number of vacancies \item
quota of the election \item stages of the counting (or computation)
\end{enumerate} Stages of the count are symbolized representations
of necessary information which one needs to know in order to
understand how the procedure of the count has exactly advanced. We
encapsulate the concept of states of the count in terms of typing
judgements with two value constructors named \textsf{NonFinal} and
\textsf{Final}. The latter one is used for expressing a terminal
state of the count where winners of the election are declared. More
specifically, a final judgement asserts the following.
\begin{definition}[Final Judgement] \begin{small} In an election,
assuming we have a quota $qu$, initial number of vacancies $st$, and
a list $\mathcal{L}$ of all candidates competing in the election,
\textsf{Final}$(w)$ is a final stage of the computation, where $w$
is the final list consisting of all of the declared elected
candidates.  \end{small} \begin{center} $qu,st,\mathcal{L}   \vdash
\mathsf{Final}(w)$ \end{center} \end{definition} Furthermore, before
reaching to a final stage in any count, the process may go through
several intermediate satges where, for example, counting of ballots
happens or some candidates are elected or eliminated. To formalize
such non-final stages, we introduce the constructor
\textsf{Non-Final}. At every non-final stage of the counting, we
need to know that if there are ballots to be counted, how the votes
have been distributed up to now, what is the tally of each
candidate, and if any candidate was elected or eliminated from the
election. Therefore, for stages of the count to thoroughly inform
the scrutineers of how the situation of the count is at the moment,
it is necessary and sufficient to contain the following components:
\begin{enumerate} \item the (possibly) uncounted ballots \item the
tally of votes of each candidate \item the set of ballots that have
been counted for each individual candidate.  \item a group of
candidates already elected who have exceeded the quota \item a group
of candidates called \emph{elected} candidates \item a group of
candidates called \emph{continuing} candidates \end{enumerate}
\begin{definition}[Non-Final Judgement] \begin{small} In an
election, assuming we have a quota $qu$, initial number of vacancies
$st$, and a list $\mathcal{L}$ of all candidates competing in the
election, \textsf{NonFinal}$(ba,t,p,bl,e,h)$ is an intermediate
stage of the computation, where $ba$ is the list of uncounted
ballots at this point, $t$ is the tally list recording the number of
votes of each candidate has received up to this point, $p$ is the
pile list of votes assigned to each candidate, $bl$ is the list of
elected whose surplus have not yet been transferred, $e$ is the list
of elected candidates by this point, and $h$ is the list of
continuing candidates up to this stage.  \end{small} \begin{center}
$qu,st,\mathcal{L}   \vdash  \mathsf{NonFinal}(ba,t,p,bl,e,h)$
\end{center} \end{definition} Definition~\ref{jud:hol} specifies the
exact HOL encoding of the sates of the computation. We should note
that the use of lists, instead of sets, for ballots, and continuing
or elected candidates is simply for the convenience of formalisation
in a theorem prover. But the counting rules ,defined shortly
afterwards, make no essential use of this representation.
Furthermore, by choosing to formalize the tally and pile as lists
rather than functions operating on the list of candidates,
judgements become an instance of the equality type class which we
use later on in specification and reasoning on counting rules.
Additionally, this formulation reduces the gap between an actual
certificate and its abstract syntactic representation which we refer
to as a \emph{formal certificate that is simply a list of
judgements}.  \begin{definition}\label{jud:hol} Assume
\HOLty{:ballots}, \HOLty{:tallies}, and \HOLty{:piles} are type
synonyms for \HOLty[-ballots,-alist]{:ballots},
\HOLty[-tallies,-alist]{:tallies}, and \HOLty[-piles,-alist]{:piles}
respectively.  \begin{holthmenv}
\HOLthm[width=80]{CheckerSpec.datatype_judgement} \end{holthmenv}
\end{definition}

As a formal certificate is constituted of judgements each
representing a state of the count, to check it for correctness, we
need to know how transition between each two consecutive judgements
in the list has occurred. According to clauses of the protocol,
there are steps that one may only legitimately take to move from one
state of the count to the next. The steps, which we call the
counting rules, are composed of some side conditions meant as
constraints on how and when one can legally apply a rule. We express
these constraints by means of auxiliary predicative assertions.
There are also implicit conditions hidden in the protocol clauses.
For example, we expect a valid certificate to have no duplication in
the list of elected or continuing candidates, and every candidate
must have only one tally and one pile at every non-final judgement.
Therefore, we need to define some auxiliary predicates to check such
subtleties implicitly existing in the protocol.  Conjunction of
these auxiliary definitions formalises individual counting rules in
HOL.


The second step is to define functions in HOL which are meant to be
the computational counterparts of the HOL specification in the step
above. Conjunction of appropriate computational definitions
constitutes the computational definition for individual counting
rules. These functions are later translated by the built-in
mechanisms of CakeML for the purpose of actual computations on an
actual certificate.


The last step is to prove equivalence of the HOL specification and
the computational assertions. By establishing this correspondence,
we demonstrate that the computational parts meet the expectation of
their specification. We therefore validate a formal certificate if
and when it is indeed valid. To show how the process concretely
proceeds, we indicate the aformentioned phases put together for the
case of \emph{elimination} rule. However, we should note that our
actual formalisation seperates the three steps to enhance
readability and clearance of the encoding.

\subsection{The Case of the Elimination Rule} There are implicit
conditions in the protocol. For example, the name of each candidate
appears only once in the initial list of competing candidates. Also
no one is assigned with more than a single pile and a unique tally.
Therefore for formalising the legislation, we define predicates that
correspond to such implicit requirements. For instance, the
predicate \textsf{Valid\_Init\_CandList} restricts legal initial
lists of candidates to those which are non-empty and have no
duplicate.  \begin{holthmenv}
\HOLthm[width=80]{CheckerSpec.Valid_Init_CandList_def}
\end{holthmenv} Moreover, at every stage of the count, every
candidate has exactly one tally and one pile. Therefore, if a
judgement in a certificate meliciously allocates more than one tally
to a single candidate, the bug is detected and the certificate
rejected as invalid.  We express this condition by the following.
\begin{holthmenv} \HOLthm[width=80]{CheckerSpec.Valid_PileTally_def}
\end{holthmenv} We also have the computational twins for both of the
predicates above. Assume the lists $t$ and $l$ are given. Then the
function \textsf{Valid\_PileTally\_dec1} decides if every first
element of each pair in $t$ is a member of $l$.  \begin{holthmenv}
\HOLthm[width=80]{Checker.Valid_PileTally_dec1_def} \end{holthmenv}
Additionally, the function \textsf{Valid\_PileTally\_dec2} determins
if each element of $l$ appears as the first component of a pair in
$t$.  \begin{holthmenv}
\HOLthm[width=80]{Checker.Valid_PileTally_dec2_def} \end{holthmenv}
W prove that the formal specification \textsf{Valid\_PileTally\_def}
enforces correctness of the function
\textsf{Valid\_PileTally\_dec1}.  \begin{holthmenv}
\HOLthm[width=80]{CheckerProof.PileTally_to_PileTally_DEC1}
\end{holthmenv}

Also the specification \textsf{Valid\_PileTally\_def} implies
correctness of \textsf{Valid\_PileTally\_dec2} witnessed by the
theorem below.  \begin{holthmenv}
\HOLthm[width=80]{CheckerProof.PileTally_to_PileTally_DEC2}
\end{holthmenv} Conversely, correctness of the computational
definition \textsf{Valid\_PileTally\_dec1} and
\textsf{Valid\_PileTally\_dec2} entails their respective
specifications. Therefore, we obtain a perfect equivalence between
the specification and computational condition for tallies and piles
to be distinctively allocated to candidates. 



Aside from the implicit conditions, \emph{elimination} has explicit
constraints to meet for its legal applications. For example, item~9
of the protocl states that \begin{small} \begin{center}
\begin{minipage}{10cm} \emph{choose the candidate with the least
amount of votes and exclude them from the list of continuing
candidates}.  \end{minipage} \end{center} \end{small} So the
difference between the continuing list of candidates in the premise
of the rule with the one in the conclusion is that the weakest
candidate is no longer a continuing candidate in the comclusion of
the rule. The predicate \textsf{equal\_except} formally asserts when
two lists are equal except for one exact element.  \begin{holthmenv}
\HOLthm[width=80]{CheckerSpec.equal_except_def} \end{holthmenv} To
compute when indeed two list match with the exception of one
element, we define the function \textsf{equal\_except\_dec}.
\begin{holthmenv} \HOLthm[width=80]{Checker.equal_except_dec_def}
\end{holthmenv} This function is proven to satisfy the specification
laid down by \textsf{equal\_except} predicate.  \begin{holthmenv}
\HOLthm[width=80]{CheckerProof.EQE_REMOVE_ONE_CAND} \end{holthmenv}
Moreover, modulo extensional equality, the function
\textsf{equal\_except\_dec} is unique.  \begin{holthmenv}
\HOLthm[width=80]{CheckerProof.EQE_IMP_REMOVE_ONE_CAND}
\end{holthmenv} The fact that the candidate removed is the weakest,
is established by another stand-alone assertion which is expressed
as a conjunct of the elimination rule's defintion. Having defined
the implicit and explicit side conditions in the definition of
\emph{elimination}, we can present the formlisation of this rule in
HOL as a predicate.  \begin{holthmenv}
\HOLthm[width=80]{CheckerSpec.ELIM_CAND_def} \end{holthmenv}

\textsf{ELIM\_CAND} is defined on three kind of input; a candidate,
a triple composed of three fixed parameters which are the quota,
vacancies, and the initial list of candidates, and two judgements
one of which is the premise and the other is the conclusion of the
rule. Each of the conjuncts corresponds to a condition specified in
the definition~\ref{elim:rule}, as we already demonstrated it for a
few cases. \textsf{ELIM\_CAND\_dec} is the computational realisation
of the \textsf{ELIM\_CAND}.  \begin{holthmenv}
\HOLthm[width=80]{Checker.ELIM_CAND_dec_def} \end{holthmenv} We
already have demonstrated how the specification of the predicative
assertions imply  correctness of the corresponding computational
parts. By drawing on these theorems we obtain equivalence of the
computational \textsf{ELIM\_CAND\_dec} with its specification
\textsf{ELIM\_CAND}. The same procedure is followed to acheive
formal specification,  computational definitions, and their
correspondence for the rest of counting rules.

\subsection{The Checker formalisation} In certified computations, to
scrutinise a certificate output by a program, one can design a
second (probably verified) program that independently performs the
computations again\cite{}. Then the result of the two could be
checked against each other for validation. On the other hand,
certificate verification can be approached in a significantly more
efficient way. Instead of redoing the computations, one can encode a
checker that validates if steps visualised in a certificate to reach
from an initial state of the computation to an end state have
correctly taken place. This approach makes the checker design
simpler than the implementation which produced the certificate. As a
result, accomplishing verification of the checker becomes
considerably easier than that of the implementation which produces
the certificates. Moreover, the checker carries computations out
faster than the original program.   




In light of the above understanding, to check a given formal
certificate for correctness, we mainly need to validate if the first
judgement of the certificate is a valid initial state of the count,
if transitions from an element in the judgement list to the next
element is legitimately doable by application of a rule, and if the
last judgement of the certificate is a legal final stage of the
computation.  As we demonstrated, rules are applicable only when
they meet their respective specification. Since rules have some side
conditions which make them distinct, this implies that at every
stage actually only one of them applies. Therefore, we specify the
correct transition-check \textsf{valid\_judgements} to be a
disjunction of the predicative counting rules and prove it matches
with the computational counterpart.  \begin{holthmenv}
\HOLthm[width=80]{CheckerSpec.valid_judgements_def} \end{holthmenv}
 
Correct end result of each instance of a computation depends on
validity of its start point.  A legitimate intial judgement is one
in which every candidate's tally is intially set to zero, their
piles are empty, and the backlog and the list of elected candidates
are both emoty as well.  \begin{holthmenv}
\HOLthm[width=80]{CheckerSpec.valid_certificate_def} \end{holthmenv}
For checking a formal certifcate we therefore first  certify if the
the certificate starts at an initial stage. Then recursively we
continue checking if transitions have happened correctly, and if the
terminating state is a final one where winners are declared. The
above specification of the checker, therefore, corresponds to the
following computational formal certificate checker.
\begin{holthmenv}
\HOLthm[width=80]{Checker.Check_Parsed_Certificate_def}
\end{holthmenv} Consequently a formal certificate is validated if
and when it is valid according to the HOL specification of
\textsf{valid\_certificate}. Since the HOL specification realises
the protocol, a formal certificate is validated if and only if it
meets the protocol's expectation.


\section{Translation into CakeML and Code Extraction}

\section{Experimental Results} We have tested our approach against
some of the past Australian Legislative Assembly elections in ACT
for years 2008 and 2012 (Figure\ref{ref;figure2})\footnote{Tests
were conducted on one core of an Intel Core i7-7500U CPU~\@
2.70GHz$\times$4 Ubuntu~16.4 LTS}. The certificates have been
produced by the Haskell program extracted from our previous
formalization of ANU\_Union STV in Coq~\cite{MKet}.  \begin{figure}
\centering
%\begin{tabu} to 0.86\textwidth {X[c] X[c] X[c] X[c] X[c] X[c]}
\begin{tabular}{|l |c |c |c |c |c|c|} \hline electoral & ballots&
vacancies& candidates& time (sec)& certificate size (MB)&year\\
\hline
%\end{tabular} \begin{tabu} to 0.86\textwidth {X[l] | X[c]  | X[c] |
%X[c] | X[c] | X[c]} \begin{tabular}{c c c c c c}
Brindabella &$63334$&$5$&$19$&$86$&$54.4$&2008\\ Ginninderra
&$60049$&$5$&$27$&$118$&83.0&2008\\ Molonglo
&$88266$&$7$&$40$&$329$&211.2&2008\\
Brindabella&$63562$&$5$&$20$&$75$&74.5&2012\\
Ginninderra&$66076$&$5$&$28$&$191$&90.1&2012\\
Molonglo&$91534$&$7$&$27$&$286$&158.7&2012\\ \hline \end{tabular}
\caption{ACT Legislative Assembly 2008 and 2012} \label{ref;figure2}
\end{figure} Moreover, we have studied behavior of our executable by
examining it on certificates output on randomly generated ballots.
There are two interesting prameters to study, namely varying the
number of ballots and varying the length of each ballot.
Figure~\ref{ref;figure3} depicts results on certificates where the
number of candidates is fixed at 20, vacancies are 5, and the length
of each ballot is 12. It appears that the complexity of the program
in the number of ballots is quadratic. Also we keep the number of
ballots, vacancies, and length of each ballot fixed at 100000, 1,
and 10 respectively, in order to inspect the effect of increase in
the length of each ballot (Figure~\ref{ref;figure4}). The complexity
of the program seems to be linear with respect to this index.
\begin{figure} \centering \begin{minipage}{.45\textwidth} \centering
\begin{tabular}{|l |c |c| c|} \hline ballots &certificate size& time
(sec)\\ \hline $400000$&$523.6$&$4224$\\ $200000$&$253.3$&$938$\\
$100000$&$131.1$&$461$\\ \hline \end{tabular} \caption{Varying
number of ballots} \label{ref;figure3} \end{minipage}
\begin{minipage}{.5\textwidth} \centering \begin{tabular}{|l |c |
c|} \hline ballot length& certificate size& time (sec)\\ \hline
$6$&$60.2$&$140$\\ $12$&$124.0$&$298$\\ $18$&$180.5$&$325$\\ \hline
\end{tabular} \caption{Varying length of each ballot}
\label{ref;figure4} \end{minipage} \end{figure}

\section{Discussion and Conclusion} \section{Outline}\label{sec:out}


Certificable / Verifiable Computation.  \begin{itemize} \item
programs produce results with correctness proofs \item much easier
than, and orthogonal to, verification \item however need to prove
that valid certificates imply valid results (soundness), maybe also
that every correct result is certifiable.  \item in particular, this
is independent of the program that produces the results.  \item
mainly used to offload computation to untrusted clients.  \item
reliability is given by verifying the \emph{checker} rather than the
program that carries out computation.  \end{itemize}

\noindent Vote counting.

\begin{itemize} \item here, we employ certified computation with a
different goal: vote counting \item difference: general public /
election authority is outsourcing computation to us.  \item state of
the art: no verifiability, black box, sometimes not even open
source.  \item this is a real problem -- e.g. NSW 2012 Griffith
\item large trust base, including to trust that the right program
has been run!  \item indeed in evoting research: concept of
``universal verifiability'' means precisely that. Is the holy grail
of any evoting system.  \item for vote counting, this is a paradigm
shift: don't verify the program, but verify the checker -- down to
machine level!  \item this happens in a fully verified / verifiable
environment \item higher guarantees by machine level verification
\end{itemize}

\noindent Our contribution.  \begin{itemize} \item Have formalised
STV (complex, preferential voting scheme used in Malta, Scotland,
etc.) in HOL \item rule based to minimise gap between law and code
\item Have designed a notion of certificate that establishes the
correctness of STV counts. Think about this more \dots \item Have
developed a verified checker \item Used CakeML to give correctness
down to machine level \item smallest possible trust base:
microprocessor, basic IO system calls \end{itemize}


\section{Single Transferable Vote and Certification}

\section{Final Theorem} \begin{holthmenv}
\HOLthm[width=80]{check_countProof.check_count_compiled_thm}
\end{holthmenv}

\begin{holthmenv}
\HOLthm[def,width=90]{check_countProg.Check_Certificate_def}
\end{holthmenv}

\section{Evaluation} \begin{itemize} \item to be done.
\end{itemize}

\section{Related Work} \begin{itemize} \item Verifiability and
electronic voting \item Verified / certified computation in general
\item CakeML stuff??

\end{itemize}

\section{Conclusion}

\bibliographystyle{splncs03} \bibliography{paper} \end{document}
% vim: ft=tex
